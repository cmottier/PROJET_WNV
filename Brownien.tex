\documentclass[a4paper,10pt]{article}
\usepackage[utf8x]{inputenc}
\usepackage{amsmath}
\usepackage{amsfonts}
\usepackage{amssymb}
\usepackage[margin=1.5cm, bottom=1.5cm]{geometry}
\usepackage{multicol}
\usepackage{graphicx}
\usepackage{multirow}
\usepackage{calc}
\usepackage{bbold}



\usepackage{pstricks-add}
\usepackage{pstricks}
\usepackage{tikz,pgf,tkz-tab}

\newcommand{\vs}[1]{\vspace{#1cm}}
\newcommand{\hs}[1]{\hspace{#1cm}}


\newcommand{\ben}{\begin{enumerate}}
\newcommand{\een}{\end{enumerate}}
\newcommand{\bit}{\begin{itemize}}
\newcommand{\eit}{\end{itemize}}
\newcommand{\R}{\mathbb{R}}
\newcommand{\N}{\mathbb{N}}
                
\usepackage{alltt}

\setlength{\parindent}{0pt}
% \setlength{\columnseprule}{0.25pt}

\begin{document}
\begin{center}

\Large Modèle brownien

\end{center}

\

\section{Le modèle brownien multivarié}

\underline{Définition} : 
Soit $(X_t)_{t\in\R_+}$ un processus aléatoire indexé par $\R_+$, à valeurs dans $\R^p$ ($p\in\N^*$).\\ On dit que $(X_t)$ est un processus à accroissements indépendants et stationnaires si, pour tous $s,t\in\R_+$, le vecteur aléatoire $X_{t+s}-X_t$ est indépendant de $X_t$ et a même loi que $X_s$.  

\

\underline{Définition} : 
Le mouvement brownien multivarié $(B_t)_{t\in\R_+}$ est l'unique processus aléatoire à accroissements indépendants et stationnaires, presque-sûrement à trajectoires continues, et tel que pour tout temps $t$ fixé, $B_t$ suit la loi $\mathcal N(0_p, tI_p)$.  

\

\begin{figure}[!h]
\centering
\includegraphics[width=6cm]{PROJET/Brownien.eps}
\caption{Simulation d'un mouvement brownien en deux dimensions}
\end{figure}



\underline{Remarques} :
\bit
\item Par définition, si $(B_t)$ est le mouvement brownien dans $\R^p$, $(B_t)=(B_t^{(1)},\ldots,B_t^{(p)})$ où les $(B_t^{(i)})$ sont $p$ mouvements browniens univariés indépendants.
\item Nous renvoyons à {\red Méléard} pour la preuve de l'existence du mouvement brownien (dans le cas univarié).
\eit

\section{Inférence dans le cas du modèle brownien}

\section{Estimation des paramètres}

Nous avons vu précédemment que :
$$Y\sim \mathcal{MN}_{n\times 2}(\mathbb 1_n\mu^T,C_n,R)$$
où $\mu\in\R^{2\times 1}$ est la position géographique de l'ancêtre commun et $R\in\R^{2\times 2}$ la matrice de covariance par lignes. 



Soit $L_n$ une matrice de Cholesky de $C_n$ (donc $C_n=L_nL_n^T$)et notons $Y^*=L^{-1}Y$ et $X^*=L_n^{-1}\mathbb 1_n$. Nous avons alors :
$$Y^*\sim \mathcal{MN}_{n\times 2}(X^*\mu^T, I_n,R)$$
c'est-à-dire : 
$$Y^*=X^*\mu^T+E,\text{ avec } E\sim \mathcal{MN}_{n\times 2}(0, I_n,R).$$
Nous sommes ici dans le cadre de la régression linéaire multivariée vue dans {\red la section...}. Remarquons que maximiser la vraisemblance de $Y^*$ revient à maximiser la ressemblance de $Y$ ({\red cf section...}). Nous allons donc pouvoir utiliser les estimateurs du maximum de vraisemblance afin d'estimer les deux paramètres $\mu$ et $R$ :

$$\begin{array}{ll}
\widehat{\mu}^T&=((X^*)^TX^*)^{-1}(X^*)^TY^*\\
&= (\mathbb 1_n^T(L_n^{-1})^TL_n^{-1}\mathbb 1_n)^{-1}\mathbb 1_n^T(L_n^{-1})^TL_n^{-1}Y\\
&= (\mathbb 1_n^TC_n^{-1}\mathbb 1_n)^{-1}\mathbb 1_n^TC_n^{-1}Y
\end{array}$$

$$\begin{array}{ll}
\widehat{R}&=\frac 1{n-1}(Y^*-X^*\widehat\mu^T)^T(Y^*-X^*\widehat\mu^T)\\
&= \frac 1{n-1}(L_n^{-1}Y-L_n^{-1}\mathbb1_n\widehat\mu^T)^T(L_n^{-1}Y-L_n^{-1}\mathbb1_n\widehat\mu^T)\\
&= \frac 1{n-1}(Y-\mathbb1_n\widehat\mu^T)^TC_n^{-1}(Y-\mathbb1_n\widehat\mu^T)
\end{array}$$

Nous obtenons alors, à partir des données du virus du Nil, les estimations suivantes : 
$$\widehat\mu^T=\begin{pmatrix}
         41.2405 & -75.3132
        \end{pmatrix}
\text{ et }
\widehat R=\begin{pmatrix}
            7.8906 & 3.3800\\
            3.3800 & 28.4436
           \end{pmatrix}$$

\underline{Remarques} : 
\bit
\item Tout le code R utilisé dans ce projet est présenté en annexe. 
\item La valeur de $\widehat \mu^T$ nous donne la position géographique estimée du cas 0 de l'épidémie. 
\eit

\section{Inférence des positions géographiques ancestrales}

Les paramètres $\mu$ et $R$ étant estimés, nous souhaitons maintenant inférer les positions géographiques des nœuds internes de l'arbre phylogénétique. Nous allons pour cela utiliser l'espérance conditionnelle.

Comme nous l'avons vu {\red dans la section...}, en notant $Z$ les positions géographiques des nœuds internes, nous avons :
$$\begin{pmatrix}
   Y\\Z
  \end{pmatrix}
\sim \mathcal{MN}(\mathbb 1_N\mu^T,C_N, R)
\hs{.5}\text{ ie }\hs{.5} \text{vec}\begin{pmatrix}
                       Y\\Z
                      \end{pmatrix}
\sim \mathcal N(\text{vec}(\mathbb 1_N\mu^T), R\otimes C_N)$$
où $N$ désigne le nombre total de nœuds ($n$ feuilles et $m$ nœuds internes en excluant la racine) et $C_N$ la matrice des temps partagés de ces différents nœuds. Ainsi, l'espérance conditionnelle de $\text{vec}(Z)$ par rapport à $\text{vec}(Y)$ est donnée par :
$$\begin{array}{ll}
\mathbb E\left[\text{vec}(Z)|\text{vec}(Y)\right]
&=\mathbb E\left[\text{vec}(Z)\right]+\mathbb Cov\left(\text{vec}(Z),\text{vec}(Y)\right)\mathbb V(\text{vec}(Y))^{-1}\left[\text{vec}(Y)-\mathbb E\left[\text{vec}(Y)\right]\right]\\
&=\text{vec}(\mathbb 1_m\mu^T)+(R\otimes C_{mn})(R\otimes C_n)^{-1}\left[\text{vec}(Y)-\text{vec}(\mathbb 1_n\mu^T)\right]
\end{array}$$
où $C_{mn}$ désigne la matrice des temps partagés entre les nœuds internes (en ligne) et les feuilles (en colonne).
Les propriétés du produit de Kronecker permettent alors de simplifier :
$$\begin{array}{ll}
   (R\otimes C_{mn})(R\otimes C_n)^{-1}
   &= (R\otimes C_{mn})(R^{-1}\otimes C_n^{-1})\\
   &= (RR^{-1})\otimes (C_{mn}C_n^{-1})\\
   &= I_2\otimes (C_{mn}C_n^{-1})
  \end{array}$$

Pour inférer les positions géographiques des nœuds internes, nous allons donc utiliser l'estimateur suivant :
$$\text{vec}(\widehat Z)=\text{vec}(\mathbb 1_m\widehat\mu^T)+(I_2\otimes C_{mn}C_n^{-1})\left[\text{vec}(Y)-\text{vec}(\mathbb 1_n\widehat\mu^T)\right]$$
Remarquons que, de manière surprenante, cet estimateur ne dépend pas de $R$. 

Nous utilisons le logiciel Evolaps afin de visualiser les estimations obtenues. La figure \ref{MB} présente le résultat obtenu.  


\begin{figure}[!h]
\centering
\includegraphics[width=12cm]{Image_MB.png}
\caption{Dynamique du virus du Nil, estimée sous le modèle brownien (en rouge les plus anciens, en bleu les plus récents)}
\label{MB}
\end{figure}





\end{document}
